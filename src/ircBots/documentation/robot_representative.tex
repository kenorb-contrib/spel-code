\documentclass{report}
\begin{document}
\chapter{Robot Representative Government}
Representatives that work for constituents, regardless of actually being elected. I say (party) in brackets, as there isn't a "platform" other than representing the constituents current opinions as best as possible.

For example, if there are at least 25 people in support of something, then can petition the House of Commons in Canada. Thus it should be acceptable to petition any lower body, such as a an MP, MPP, county, mayor, or district representative.

The idea is that the robot representative would lobby the current government representatives, on behalf of it's constituents. It can do so via email, phones, letters, and physical presence at law drafting. This would in fact be a service to those representatives, as it would notify them about their constituents' live opinion on whatever topic it is.

Historically small parties get rather few votes, such as around 100, this is about as many as one candidate could realistically form a personal connection with -- all their friends.   In order to get into the thousands range, you need friends of friends to be involved in campaigning, and-or some kind of advertising.

Though for the robot representative party, even a hundred or so people would be enough.  Don't even need to have an official election to start working. The Secure Electronic Signature Regulation [1], says that a cryptographic hash of the document is an acceptable electronic signature.

So for instance a GNUPG signature should suffice, for instance if the constituent sent an email reply with the bill contents, the PGP hash from it could be used as their signature, and could be verified.  The hardest part would be making this easy for the constituents to use.

\section{App}

Can have a mobile and desktop app, which people use to view, deliberate and vote on various petitions.

\section{liquid democracy}

As it stands, liquid democracy isn't supported by the current definition of a petition, however if we have at least 25 regular signers, can likely include how many additional votes they may represent via the liquid democracy approach.

Liquid democracy is when someone delegates their vote to someone else, who generally shares their opinions, or whom they trust to make a wise choice. This may be someone in their social group who is interested in politics, and has the time to read through the various bill proposals, and sort out whether to vote for or against, perhaps even coming up with potential amendments that would make a for vote viable.

This is of course retractable, if for instance you want to vote on something yourself, or you feel someone else is better suited to represent your interests.  It may even be encouraged to select a liquid democratic representative, so your vote doesn't go to waste, by sending a reminder every parliamentary session (approximately every year or two).

For me, I might select some keywords for the kind of legislation I want to personally suggest amendments on, and vote on. While otherwise delegating to someone that holds views of the Pirate and Green Party but is pro-Israel. 

henceforward reference to voters will include liquid-democratic representatives, who may count as multiple voters due to representing other people.

\section{Bill information}

The next major issue, is getting the actual bills which are up for proposal.

Openparliament.ca seems to be a good source for federal bills, it seems to have a fairly simple layout, so it should be viable to harvest the data from it. ontla.on.ca while having slightly more complex interface is also navigable and has accessible bills.

\section{Live Presence}

If a certain riding has enough campaign contributions, it can afford to have a physical robot, which can attend committee meetings open to the public, and both record them as well as voice the opinions of the members who have logged in for the committee meeting.

For instance if a quorum (25) of voters vote to go to the meeting, then the robot will go to the meeting.  At the meeting if a majority of the logged in voters wish to say something, then the robot will say it at the next given opportunity.  For instance there could be some kind of LED that lights up, or a hand that is raised, which the facilitator can acknowledge when there is time for the robot to speak.

If using state of the art technologies of today, could have a 360 degree recording, so people could don their virtual reality glasses and be virtually present at the meeting. When looking down where the robots body would be, a person would see the console or chat interface of voters, so can work on amendments and voting for what the robot will say next.

For accessibility there can be a chat only interface, and the facial and body expressions of the various people in the room could be added to the chat via feature recognition software.


When the robot speaks, it could be of the form "representing <x number of voters>, <statement>". That way the people at the committee meeting would know how much weight the statements carry.

In the transitional period, where there isn't a robot that can get itself to a committee meeting, a person could substitute,  though they would have to be paid enough to cover their time, and ideally would have a live stream video connection on their laptop.

Then the person could raise their hand, and say what the majority of those logged in wish to have said.
Ideally this would also be someone who has many liquid democracy votes, as if it is above quorum (25) and the majority at the meeting then they could speak on behalf of all the voters present.

\section{Conclusion}

In the same fashion, it is actually possible for a robot to do the work of a political representative.  Or at least the work that is accessible to any member of the public, though which most members of the public don't pursue as they have busy lives -- like attending committee meeting.

If I could virtually attend the local municipal, provincial, and-or federal committee meetings virtually I would. At least those of them which are important to me -- even though some of them are hundreds or thousands of kilometers away from me.  Of course I would also be happy if there was some app or email service, which allowed me to vote yes or no to on real bills being passed right now, and then have my opinion matter enough to be included on a petition to the local representative.

Eventually if it works well enough, these robot representatives could go all the way to the top.  Though I understand MP's and MPP's have a variety of tasks which are not policy related, those could also be automated if sufficiently described.

[1] Secure Electronic Signature Regulation http://laws-lois.justice.gc.ca/eng/regulations/sor-2005-30/page-1.html#h-2
\end{document}
